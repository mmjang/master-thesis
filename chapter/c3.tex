\chapter{地下探测FDTD仿真技术研究}
\section{天线模型仿真}
\subsection{FDTD天线模型原理}
\subsection{仿真实例}
\section{拟真土壤模型}
\subsection{拟真土壤模型原理}
Peplinski 等人提出一种模拟真实土壤电磁特性的半经验模型。在此模型中,土壤由三种成分构成,分别是:
颗粒直径在0.05mm到2.0mm之间的沙土,颗粒直径在0.002mm到0.05mm之间的粉土,以及颗粒直径在0.002
以下的粘土。这三种成分的配比不同可表示不同的土壤类型,进而也影响到其电磁特性。同时,土壤中含水量也对
其电磁特性起重大影响。综合这些因素,土壤的复介电常数$\epsilon_m$由式\ref{eqn:peplinskiall}给出。
\begin{equation} 
	\label{eqn:peplinskiall} 
	\begin{aligned} \epsilon_{m} &=\epsilon_{m}^{\prime}-j \epsilon_{m}^{\prime \prime} \\ 
\epsilon_{m}^{\prime} &=1.15 \left[1+\frac{\rho_{b}}{\rho_{s}}\left(\epsilon_{s}^{\alpha}\right)+m_{v}^{\beta^{\prime}} \epsilon_{f w}^{\alpha}-m_{v}\right]^{1 / \alpha} - 0.68\\ 
\epsilon_{m}^{\prime \prime} &=\left[m_{v}^{\beta^{\prime \prime}} \epsilon_{f w}^{\prime \prime \alpha}\right]^{1 / \alpha} \end{aligned}
\end{equation}

上式中,$\epsilon_{m}^{\prime}$ 与 $\epsilon_{m}^{\prime}$分别是复介电常数
$\epsilon_m$的实部与虚部;$\rho_{b}$ 与 $\rho_{s}$分别是土壤的总密度和其中沙土成分
的密度(单位:$g/cm^3$);$\beta^\prime$与$\beta^{\prime \prime}$分别是与土壤构成
相关的常数,其表达式为\ref{eqn:soil_beta},其中$S$与$C$分别是沙土与粘土的构成比例
($0<S<1, 0<C<1$)。
\begin{equation} 
	\label{eqn:soil_beta}
	\begin{aligned}
\beta^{\prime}&=1.2748-0.519 S-0.152 C \\
\beta^{\prime \prime}&=1.33797-0.603 S-0.166 C
	\end{aligned}
\end{equation}

式\ref{eqn:peplinskiall}中$\epsilon_{f w}^{\prime}$与$\epsilon_{f w}^{\prime \prime}$
分别是土壤中自由水的实部与虚部,其表达式由式\ref{eqn:soil_water}给出。
\begin{equation} 
	\label{eqn:soil_water}
	\begin{aligned}
		\epsilon_{f w}^{\prime}&=\epsilon_{w \infty}+\frac{\epsilon_{w 0}-\epsilon_{w \infty}}{1+\left(2 \pi f \tau_{w}\right)^{2}} \\
		\epsilon_{f w}^{\prime \prime}&=\frac{2 \pi f \tau_{w}\left(\epsilon_{w 0}-\epsilon_{w \infty}\right)}{1+\left(2 \pi f \tau_{w}\right)^{2}}+\frac{\sigma_{\mathrm{eff}}}{2 \pi \epsilon_{0} f} \frac{\left(\rho_{s}-\rho_{b}\right)}{\rho_{s} m_{v}}
	\end{aligned}
\end{equation}
其中,$\epsilon_{w \infty}$是$\epsilon_{f w}^{\prime}$在高频时的极限;$\tau_{w}$为自由水的
弛豫时间常数;$\epsilon_{w 0}$为水的静态相对介电常数,其值为80.1;$\sigma_{\mathrm{eff}}$为有效电导率
,由经验公式\ref{eqn:soil_eff}给出。
\begin{equation}
	\label{eqn:soil_eff}
\sigma_{\mathrm{eff}}=0.0467+0.2204 \rho_{b}-0.4111 S+0.6614 C
\end{equation}
\subsection{建模实例}
\section{B 扫数据的批量生成}
\section{本章小结}
本章首先研究了时域积分方程时间步进算法的阻抗元素精确计算技术,分别采用DUFFY变换法与卷积积分精度计算法计算时域阻抗元素,通过算例验证了计算方法的高精度。

