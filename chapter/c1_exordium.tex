\thesischapterexordium

\section{研究工作的背景与意义}

探地雷达(GPR)使用电磁场探测有损介电材料,以检测介质内内的结构和材料特性的变化。
迄今为止,大多数应用都采用天然地下介质,但也会出现在人造复合材料如混凝土,沥青和其他建筑材料中的广泛应用。在这种有损介电材料中,电磁场在被吸收之前会渗透到某个深度。
对于GPR,电磁场作为基本上非色散的波传播。发射的信号穿过介质,被阻抗的变化散射或反射,产生类似于发射信号形状的回波,通过雷达波组成的B扫图像便可以观察到地下结构或者目标物的特征。GPR平台通常工作在1 MHz至1000 MHz的频率范围内。在低于1M的频率下,电磁波具有较大的散射特性,这种工作频率下一般称为电磁感性探测法。在较高频率下,信号在地层中的衰减较大,使得穿透极为有限。在过去的三十年来[3],探地雷达广泛应用于地质探测、管道勘查、遗迹探查、道路和隧道勘探、扫雷和爆炸物探测等领域。

目前,探地雷达主要采用瞬态脉冲体制。瞬态脉冲雷达具有较宽的工作带宽,而且探地雷达
所面对的是具有非均匀性、强衰减性及色散效应的地下有耗介质,电磁波传播环境复杂多变。
因此探地雷达接收回波中会有较多的杂波干扰,给后期的数据解释工作带来困难[4]。已有
的探地雷达数据处理方法主要集中在杂波抑制与图像重建两个方面,常见方法有噪声抑制
、时变增益、背景去除、滤波、反卷积、基尔霍夫偏移、Stolt偏移等等[5]。这些方法的
使用,需要探地雷达研究人员根据不同的工作环境特点与目标特性,选择合适的方法与合
适的算法参数以获得理想的处理效果,对研究人员的要求较高。在某些复杂的工作环境下
,这些传统方法甚至可能完全达不到预期效果。因此,开发一种对算法使用人员要求较低
,能在不同的使用环境下自动提取探地雷达回波信号中关键信息并由此对目标位置、深度
、大小、材质等关键参数进行识别的探地雷达处理算法是极其有必要的。

近年来,随着硬件计算能力的增强和研究的不断深入,机器学习领域已发展至深度学习阶段[6]
。深度学习网络模型已经开始广泛应用于普通的图像的识别和处理。在图像识别中采用深度学习
能大大提高其准确性,并且耗时短,从而大大提高了计算效率[7]。探地雷达的B扫数据是
由实数组成的二维数组,和数字图像具有相似性。特征提取是雷达数据处理中的一个重要研究
内容,而深度学习可以更为有效地对特征进行分层提取。深度学习可以对输入信号进行分层特
征提取,能更好地使用特征表达原始输入,并且这些特征集合代表了原始数据中不同层次的抽
象概念及意义。因此,将深度学习相关方法应用到探地雷达目标识别具有广阔的应用前景。

\section{国内外研究历史与现状}
% 机器学习是人工智能的一个分支,是人工智能最为常见的一种实现方法。一系列学习算法被应用,能够在大批量历史数据中进行学习并得出相应的规律,并为后续的研究工作奠定理论基础。
% 自二十世纪九十年代至今,机器学习的发展共出现了两次比较大的变革浪潮,这两次比较大
% 的浪潮分别是浅层学习以及深度学习。20 世纪 90 年代末期,BP神经网络算法给机器学习
% 提供了一个新的研究领域,并掀起了浅层学习的热潮。2006年,世界上最具有权威之一的
% 学术期刊《Science》上刊登了一篇关于深度学习的文章,作者是加拿大多伦多大学教授
% 辛顿(Hinton)和他的两个门徒,这篇文章给当时的研究指明了新的方向,自此开启了
% 机器学习的第二股浪潮[8]。随着研究工作的深入以及研究力度的增大,斯坦福大学,
% 蒙特利尔大学等世界知名高校纷纷加入到了这个研究行列中。同时,各个国家政府也开
% 始纷纷斥资于深度学习的研究中,2010年美国国防部成为第一个资助该项目的政府机构
% ,这一举措对整个深度学习领域都具有深刻的意义。2010年至今,深度学习在各个领域
% 的应用也越来越广泛,Google和微软研究院的研究者通过卷积神经网络算法减低了语音
% 识别错误率百分之二十到百分之三十,取得了十几年语音识别研究以来最突出的研究成果
% 。时隔两年卷积神经网络技术又有了新的研究成果,在ImageNet问题[10]上使用该项
% 技术以后,错误率由原来的26%降至15%。而在大数据技术开始盛行的今天,各个国家
% 的知名公司也都开始向深度学习的研究中投入一定的资源,抢占深度学习技术的先机。
% 图像处理是深度学习的第一个尝试的领域。1989 年卷积神经网络(CNN)的提出[11],
% 给图像识别指明了研究的方向,虽然它一直没有取得巨大的突破,但在单一方面取得了不
% 错的绩效,研究员将 CNN 应用于手写数字中取得了当时最好的研究效果[12]。CNN 处
% 理规模庞大的图像处理中的效果稍微有所欠缺,因此一直没有收到重视。直至辛顿(Hinton)
% 和他的门徒通过更深层的卷积神经网络在 Image Net 问题上研究出了当时世界上最杰出的
% 成果,从而改变了此现状[13]。在辛顿(Hinton)构建的卷积神经网络深度模型中,将图像
% 像素作为初始输入,并且使用无监督逐层学习方法来学习特征。近年来,深度学习网络模型
% 已经开始广泛应用于普通的图像的识别和处理[14]。在图像识别中采用深度学习能大大提高
% 其准确性,并且耗时短,从而大大提高了计算效率。从深度学习的发展以及图像处理技术的
% 发展形势来看,深度学习将逐渐成为图像处理方法的主流。
\subsection{探地雷达发展历史}

\subsection{深度学习研究历史}
深度学习作为机器学习的一个分支,采用算法处理数据并模仿思维过程,或开发抽象。深度学习使用多层算法来处理数据,理解人类语音并在视觉上识别对象。信息通过每一层传递,前一层的输出为下一层提供输入。网络中的第一层称为输入层,而最后一层称为输出层。两者之间的所有层都称为隐藏层。每层通常是包含一种激活函数的简单统一算法。
特征提取是深度学习的另一个方面。特征提取使用算法自动构建数据的有意义的“特征”,以用于训练,学习和理解。

深度学习的历史可以追溯到1943年,当时Walter Pitts和Warren McCulloch创建了一个基于人类大脑神经网络的计算机模型。他们使用算法和数学的组合,他们称之为“阈值逻辑”来模仿思维过程。从那时起,深度学习已经稳步发展,其发展经历了两次重大突破。

开发深度学习算法的最早努力来自Alexey Grigoryevich Ivakhnenko 和Valentin Grigor'evich Lapa。1965年,他们使用具有多项式激活函数的模型,然后进行统计分析。每一层都会将最佳统计选择的特征转发到下一层。

Kunihiko Fukushima使用了第一个“卷积神经网络”。Fukushima设计了具有多个汇集和卷积层的神经网络。 1979年,他开发了一种名为Neocognitron的人工神经网络,该网络采用分层的多层设计。这种设计允许计算机“学习”识别视觉模式。这些网络类似于当代的网络结构,经过多层重复激活的强化策略训练,随着时间的推移网络性能逐渐增强。此外,Fukushima的设计允许通过增加某些连接的权值来手动调整重要功能。

反向传播算法,及其在深度学习模型的训练的在应用,在1970年开始得到长足发展。1985年,Rumelhart,Williams和Hinton证明在神经网络中的反向传播可以学习到数据的分布特征。1989年,Yann LeCun在贝尔实验室提供了第一次反向传播的实际演示。他将卷积神经网络与反向传播结合到读“手写”数字上。该系统最终用于读取手写支票的数量。

深度学习的下一个重要里程碑发生在1999年,当时计算机开始变得更快处理数据并开发了GPU(图形处理单元)。使用GPU处理图片的处理速度更快,在10年的时间内,计算速度提高了1000倍。在此期间,神经网络开始与支持向量机竞争。虽然神经网络与支持向量机相比可能较慢,但神经网络使用相同的数据提供了更好的结果。随着更多训练数据的添加,神经网络还具有继续改进的优点。

大约在2000年,研究者发现“梯度消失”问题。即在较低层中形成的特征没有被上层学习,因为没有学习信号到达这些层。这是基于梯度的学习算法所存在的问题。问题的根源被证明是某些激活函数引起的。许多激活函数限制了它们的输入,从而减小了它们的输出范围。这产生了在极小范围内映射的大范围的输入。在这些输入范围内,大的变化将减少到输出的微小变化,导致梯度消失。用于解决该问题的两种解决方案是逐层预训练和长短期记忆结构的开发。

2009年,斯坦福大学教授Fei-Fei Li发起了ImageNet,组建了一个包含超过1400万张标记图像的免费数据库。互联网充满了未标记的图像。需要标记的图像来“训练”神经网络。李教授说:“我们的愿景是大数据会改变机器学习的方式。数据驱动学习。”

到2011年,GPU的速度显着提高,从而可以使训练卷积神经网络无需逐层预训练。随着计算速度的提高,深度学习在效率和速度方面具有显着优势。其中一个例子是AlexNet,这是一个大型卷积神经网络,其架构在2011年和2012年赢得了多项国际比赛。

\subsection{机器学习以及深度学习在探地雷达中的应用}
机器学习在探地雷达信号处理中的早期应用主要基于较简单的浅层神经网络或支持向量机模型
。2001年,王群等人将前向神经网络应用于探地雷达地雷探测识别,实现对干扰物信号与地
雷信号的分离[15]。2004年,刘敦文等人运用人工神经网络理论和方法,建立了用于隧道衬
砌厚度探地雷达探测信号解释的BP神经网络模型,对某公路隧道衬砌检测厚度进行了分析应
用,提高了探地雷达信号解释精度和工作效率[16]。2005年,电子科技大学胡进峰等人,将
多目标识别支撑矢量机与探地雷达目标识别相结合,得到了基于一对一支撑矢量机的探地雷
达多目标识别方法[17]。

近期,深度学习在探地雷达信号处理方面的应用开始见于文献。2014年,美国Besaw等人
提出使用深度信念网络(DBN)对探地雷达数据进行处理[18]。此文献中,DBN首先以无监
督学习算法预训练,以此获得输入数据的压缩化表示并作为特征识别器。然后,DBN又被标签
数据监督学习,以此获得预测模型。此DBN模型成功达到百分之91的预测正确率和百分之1.4
的虚警率。同年,西班牙Núñez-Nieto等人将2.3GHz和1GHz的MALA商业探地雷达所采集的
数据使用深度感知机进行监督训练,并与传统线性回归方法做比较,取得良好的效果[19]。
2017年,Wei Wang等人利用深度自编码器识别墙后人体目标,并用降噪编码器进一步提高
特征表示的效率[20]。

从上述研究进展可看出,目前机器学习特别是深度学习已在探地雷达数据处理领域得到初步
尝试。但是,绝大部分已有研究将重点放在目标有无的识别上,还没有对目标的几何特征
介质特征做进一步的探索。因此,此论文开展的研究具有较强的现实意义。
\section{本文的主要贡献与创新}
本论文以探地雷达地下目标的特征识别为主线,主要创新点与贡献如下:

1. 将土壤的介质参数模型和分形理论结合起来,实现拟真土壤模型的算法生成。

2. 构思出一系列建模步骤,实现将实际天线模型融入FDTD网格仿真过程。

3. 基于卷积神经网络实现对地下目标的特征识别。
\section{本论文的结构安排}
本文的章节结构安排如下:

第二章研究深度领域最核心的基础概念和理论。梳理了机器学习相关概念包括感知器和激活函数的数学意义和
提出背景;在介绍前馈神经网络的基础上着重阐述反向传播算法的原理和流程;最后本章引入卷积神经网络这种
网络结构,为本文后面章节的研究提供理论基础。

第三章研究探地雷达地下目标识别的仿真技术。通过拟真土壤模型的生成算法和真实天线模型的构建,并结合
地下目标模型的批量生成技术得到大量与实际数据相似的雷达B扫图像,为下一章深度学习算法的性能验证
提供数据基础。

