\chapter{全文总结与展望}

\section{全文总结}
本文以探地雷达地下目标识别这个问题为背景,针对目前传统算法需要大量的人工干预和专业知识
的局限性,并结合探地雷达B扫数据和光学图像的相似性,使用深度学习技术来解决地下目标
识别的问题。本文在近年来国内外机器学习在地下目标识别的研究工作的基础上,不仅做到了
对地下目标有无的识别,还发展出了对地下目标介质、尺寸、深度等信息的识别方法。本文所形成的
目标识别方法最终在仿真数据和实测数据上均得到验证。

概括来说本文研究过程中的主要难点在于:

1. 在构建拟真土壤模型时。查阅前人关于土壤的组成和含水率对复介电常数的影响的研究成果,
并将其与分形特征数据生成算法相结合,形成拟真土壤模型的建模方法。

2. 在天线建模与模型批量生成方法研究过程中。研究python语言在探地雷达地下目标识别建模领域
的应用原理,形成一套自动化地下目标仿真程序。

3. 在构建深度网络模型时。形成一组合适的B扫数据的预处理流程。比较各种参数和网络结构对识别效果
的影响,并在最后得到准确率优秀的网络模型。

\section{后续工作展望}
深度学习领域在最近几年发展迅速。2018年,关于深度学习在电磁探测领域的研究的文献更是相继出现。
在本文研究工作的基础上,后续工作可在以下几个方面继续展开:

1. 本文在仿真数据上实现了对地下目标材质、位置、深度、尺寸等信息的识别,但是由于实验条件所限,
在实验数据上并没有来得及对材质深度等信息的识别进行验证。所以下一步的工作首先可就本文的方法在
实验数据上的全面验证上展开。

2. 传统的电磁反演方法主要是基于电磁正演与梯度下降优化求得目标的轮廓和介电常数分布。其最大的缺点
是:一、电磁正演需要消耗大量的计算时间;二、要取得理想的反演效果,正演模型需要与实际模型高度吻合。
深度学习由于其模型结果的输出速度快,可以拟合复杂非线性问题的特点,有望实现通过雷达B扫图像反推
地下目标的真实形状与电磁参数。

3. 本文在进行天线建模过程中,由于整个FDTD网格的空间步长是一致的,所以在对天线几何模型的细节处理方面
存在妥协,因而影响了天线仿真的精度。若使用次网格等技术,则有望实现在模型细节之处对FDTD网格进一步局部剖分,
解决模型精度与计算速度之间的矛盾。