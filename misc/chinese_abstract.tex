\begin{chineseabstract}
探地雷达是使用电磁波探测地下环境从而检测介质内结构和材料特性的变化的一种方法。地下目标衰减大等问题会导致回波中会有较多的杂波干扰,给后期的数据解释和目标识别工作带来困难。
本文利用深度学习在特征提取方面的优势,并结合探地雷达信号与数字图像的相似性,将深度学习运用到探地雷达目标识别问题。针对已有研究成果的不足之处,本文提出了一套基于数据分割的雷达图像预处理流程及标签生成方法并在此基础上通过设计深度神经网络完成了对目标位置、介质、尺寸和深度的识别,填补了已有文献只对目标有无进行识别的空白,最终正确率达到90\%左右。此外,本文还探索了基于土壤介质参数模型和真实天线模型的建模仿真技术,最终得到了训练所需的拟真雷达数据。

本文简要阐述了深度学习领域相关基础概念。感知器是用来模拟生物体神经元活动的模型,感知器的级联可用来表征复杂的决策过程。人工神经网络一般由多层感知器级联而成。神经网络的误差函数可在交叉熵的基础上定义,误差函数用来在采用梯度下降算法优化网络参数时评估模型预测性能。反向传播算法使计算误差函数对于各层权值和偏置的梯度成为可能。卷积网络是解决图像等二维数据识别问题的有效网络结构,适用于雷达B扫图像的相关识别问题。

本文研究了FDTD仿真的基本原理并基于开源数值仿真平台gprMax、土壤介质参数半经验模型和FFT分形数据生成算法形成真实土壤模型构建方法;针对模拟天线传输特征的问题,研究将真实天线几何模型内置到FDTD网格的方法,并在一个实际算例中完成了从天线设计稿到FDTD网络模型的转换。本文还相应提出基于配置文件模板和调度程序的拟真雷达B扫图像批量生成方法。

本文研究了探地雷达B扫数据预处理方法及其数学描述,形成以水平分割为基础的数据分解及标签设置方法;对于位置、介质和尺寸的识别,本文设计并训练了一个深度卷积神经网络用来对分割数据的介质种类做多分类识别,并以此为基础完成对位置和尺寸信息的推算;对于深度信息的识别问题,本文构建基于回归的深度神经网络,并直接在上一个网络的识别基础上给出目标的深度信息。最后,本文分别在仿真数据和实测数据上验证了模型的识别效果。结果表明深度学习在探地雷达目标识别领域具有广阔应用前景。

\chinesekeyword{探地雷达,目标识别,FDTD仿真,深度学习,卷积神经网络}
\end{chineseabstract}

