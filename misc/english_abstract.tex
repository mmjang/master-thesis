
\begin{englishabstract}
	Ground penetrating radar is a method of detecting changes in the structure and material properties in a medium by using electromagnetic waves to detect the underground environment. Problems such as large attenuation of the underground target will cause more clutter interference in the echo, which will bring difficulties to the later interpretation of the data. The existing data processing method of ground penetrating radar requires researchers to select the appropriate method and appropriate algorithm parameters according to different working environment characteristics and target characteristics to obtain the desired processing effect, and the requirements of researchers are high. In recent years, deep learning network models have begun to be widely used in the recognition and processing of ordinary images. The use of deep learning in image recognition can greatly improve its accuracy, and it takes a short time, which greatly improves the calculation efficiency. The B-scan data of the ground penetrating radar is a two-dimensional array composed of real numbers, which has similarities with digital images. This paper propose a underground target recognition method based on deep learning. The method adopts a network structure based on convolutional neural network, which realizes the recognition of underground target position, medium, size and depth and achieves excellent recognition accuracy. 
	
	This thesis examines the basic concepts related to deep learning. A perceptron is a model used to simulate the activity of a living body's neurons, and a cascade of perceptrons can be used to characterize complex decision-making processes. Artificial neural networks are generally cascaded by multilayer perceptrons. The error function of the neural network can be defined on the basis of cross entropy, which is used to evaluate the model prediction performance when using the gradient descent algorithm to optimize the network parameters. The backpropagation algorithm makes it possible to calculate the error function for each layer of weights and offsets. The convolutional network is an effective network structure for solving two-dimensional data recognition problems such as images, and is suitable for solving the recognition problem of radar B-scan images. 
	
	This theis studies the basic principles of FDTD simulation and forms a real soil model construction method based on open source numerical simulation platform Gprmax, a soil medium parameter semi-empirical model and FFT-based fractal data generation algorithm. For the problem of analog antenna transmission characteristics, the real antenna geometry model is built into the FDTD grid. Finally, the pseudo-real radar B-scan image is obtained in batches in combination with the configuration file template and the scheduler. 
	
	In this paper, the method of pre-processing of ground penetrating radar B-scan data and its mathematical description are studied, and the data decomposition and label setting method based on horizontal segmentation is formed. For the identification of position, medium and size, a deep convolutional nerve is designed and trained. The network is used to identify the types of media, and based on this, the calculation of position and size information is acquired. For the identification of depth information, this work constructs a deep neural network based on regression and the depth information of the target is given based on the identification result of the previous network. Finally, this thesis verifies the recognition performance of the model on the simulation data and the measured data.
	
	\englishkeyword{ground penetrating radar, target recognition, 
	FDTD simulation, deep learning, convolutional neural networks}
\end{englishabstract}


