
\begin{englishabstract}
	Ground penetrating radar is a method of detecting changes in the structure and material properties of a medium using electromagnetic waves to conduct subsurface environment surveys. Problems such as large attenuation of the underground target will cause significant clutter interference in the echoed signals, which will bring difficulties to the later data interpretation and target recognition. In this paper, the advantages of deep learning in feature extraction are combined with the similarity between ground penetrating radar signals and digital images. Deep learning is applied to solve the target recognition problems of ground penetrating radar. Aiming at the shortcomings of the existing research results, this paper proposes a radar image preprocessing process based on data segmentation, and on this basis, the position, medium, size and depth of targets are identified, which fills the void that existing literature mainly focuse on targets' position. The final identification accuracy rate reaches 90\%. In addition, this paper also explores the modeling and simulation technology based on the dielectric properties model of soil and the real antenna model, and finally obtains the pseudo-real radar data.
	
	This thesis illustrates the basic concepts related to deep learning. A perceptron is a model used to simulate the activity of a living body's neurons, and a cascade of perceptrons can be used to characterize complex decision-making processes. Artificial neural networks are generally cascaded by multilayer perceptrons. The error function of the neural network can be defined on the basis of cross entropy, which is used to evaluate the model prediction performance when using the gradient descent algorithm to optimize the network parameters. The backpropagation algorithm makes it possible to calculate the error function for each layer of weights and offsets. The convolutional network is an effective network structure for solving two-dimensional data recognition problems such as images, and is suitable for solving the recognition problem of radar B-scan images. 
	
	This theis studies the basic principles of FDTD simulation and proposes a real soil model construction method based on open-source numerical simulation platform gprMax, a soil dialectric semi-empirical model and FFT-based fractal data generation algorithm. For the problem of simulating antenna transmission characteristics, the geometry models of real antennas are built into the FDTD grid. Finally, the pseudo-real radar B-scan image is obtained in batches in combination with the configuration file template and an scheduler based on python. 
	
	In this paper, the method of pre-processing of ground penetrating radar B-scan data and its mathematical description are studied, and the data decomposition and label setting method based on horizontal segmentation is formed. For the identification of position, medium and size, a deep convolutional neural network is designed and trained. The network is used to identify the types of media, and based on this, the calculation of position and size information is acquired. For the identification of depth information, this work constructs a deep neural network based on regression and the depth information of the target is given based on the identification result of the previous network. Finally, this thesis verifies the recognition performance of the model on the simulation data and the measured data. The results show that deep learning has broad application prospects in the field of ground penetrating radar target recognition.
	
	\englishkeyword{ground penetrating radar, target recognition, 
	FDTD simulation, deep learning, convolutional neural networks}
\end{englishabstract}


