
\thesisacknowledgement
在电子科大三年的研究生求学经历马上就要结束了。我首先要衷心感谢我的导师赵青教授对三年来学习和生活上的帮助。无论是在刚入学时科研方向和学习计划的选择,还是在正式进入课题组后科研项目上出现困难时的解决途径,抑或是论文选题的切入点等方面,赵老师都给予了细致和全面的指导。赵老师高远的科研追求,勤勉的工作态度,还有持久的锻炼习惯令人印象深刻,也值得我向他学习。

本文的研究和写作也离不开教研室其他老师和同学的帮助。感谢刘述章教授和马春光讲师对论文的选题和具体研究工作的指导。刘老师在科研工作上深厚的学术功底和严谨的治学态度给我很大影响。感谢霍建建博士在数值仿真和信号处理方面的帮助。感谢谢龙昊博士在神经网络的建模和调优方面提供的很多建设性意见。感谢王尚和王汉章硕士在现场实验时的辛勤工作。感谢王立凯硕士在雷达信号处理领域留下的宝贵研究基础。此外还要感谢同门徐文强、陈宗和张雷军硕士以及其他老师同学三年来在学习和生活上的交流与陪伴。

最后,感谢各位评审专家的耐心评阅和指导。